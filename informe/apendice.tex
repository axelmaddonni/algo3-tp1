\subsection{Demostración del Lema 3.1}

Vamos a demostrar que $T(n) = F(n)$, si $n=2k$ con $k\in\mathbb{N}$, por inducción en $k$.

\textbf{Caso base:} $k=0 \implies n=0$

  $T(0) = 1$ 

  Dado que las sumatorias y las productorias en las que el límite inferior es más grande que el superior se devuelve el elemento neutro para la suma y el producto, 0 y 1 respectivamente, tenemos que $F(0) = 1$. Entonces demostramos el caso base.

\textbf{Paso inductivo:} Supongamos que $T(n) = F(n)$. Veamos que también vale $T(n + 2) = F(n + 2)$ \\
  \begin{equation}
  \label{eq:ind3-1}
  \begin{aligned}
  T(n+2) &= (n+2)^3 + (n+2)^2 T(n) \\
         &= (n+2)^3 + (n+2)^2 F(n) \\
         &= (n+2)^3 + (n+2)^2 \sum_{i=0}^{\frac{n}{2}-1} \left((n-2i) \prod_{j=0}^{i}(n-2j)^2\right) + (n+2)^2 \prod_{i=0}^{\frac{n}{2}-1}(n-2i)^2\\
         &= (n+2)^3 + \sum_{i=0}^{\frac{n}{2}-1} \left((n-2i) (n+2)^2 \prod_{j=0}^{i}(n-2j)^2\right) + (n+2)^2 \prod_{i=0}^{\frac{n}{2}-1}(n-2i)^2
  \end{aligned}
  \end{equation}

  Por otro lado tenemos

  $F(n+2) = \sum_{i=0}^{\frac{n}{2}} \left((n+2-2i) \prod_{j=0}^{i}(n+2-2j)^2\right) + \prod_{i=0}^{\frac{n}{2}}(n+2-2i)^2$

  Ahora hagamos un par de observaciones:

  \begin{equation}
  \label{eq:ind3-2}
  \begin{aligned}
  (n+2)^2 \prod_{j=0}^{i}(n-2j)^2
  &= (n+2)^2 \prod_{t=1}^{i+1}(n-2(t-1))^2\\
  &= (n+2)^2 \prod_{t=1}^{i+1}(n+2-2t))^2\\
  &= \prod_{t=0}^{i+1}(n+2-2t)^2 
  \end{aligned}
  \end{equation}

  \begin{equation}
  \label{eq:ind3-3}
  \begin{aligned}
  (n+2)^3 + \sum_{i=0}^{\frac{n}{2}-1} \left((n-2i) \prod_{t=0}^{i+1}(\hdots)\right)
  &= (n+2)^3 + \sum_{r=1}^{\frac{n}{2}} \left((n-2(r-1)) \prod_{t=0}^{r}(\hdots)\right) \\
  &= (n+2)^3 + \sum_{r=1}^{\frac{n}{2}} \left((n+2-2r)) \prod_{t=0}^{r}(\hdots)\right) \\
  &= \sum_{r=0}^{\frac{n}{2}} \left((n+2-2r)) \prod_{t=0}^{r}(\hdots)\right)
  \end{aligned}
  \end{equation}

  Notar que la igualdad \ref{eq:ind3-2} puede instanciarse para $i = \frac{n}{2}-1$. 
  Aplicando las igualdades \ref{eq:ind3-2} (dos veces) y \ref{eq:ind3-3} sobre la ecuación \ref{eq:ind3-1}, llegamos a que $T(n + 2) = F(n + 2)$. $\qed$


\subsection{Kamehameha: el caso promedio se parece al peor caso}

Formalicemos que queremos decir con esto. Supongamos que los puntos de nuestro \emph{input} provienen, al azar, del conjunto $\mathcal{X}_M = \{1, \cdots., M\} \times \{1, \cdots, M\}$. Lo que probaremos aquí es que si M es grande, entonces la probabilidad de que necesitemos $\frac{n}{2}$ rectas para cubrir todos los puntos (peor caso) es alta.

Primero, calculemos la cantidad de formas de elegir $n$ puntos de ese conjunto:

\[|\{\text{formas de elegir $n$ puntos de $\mathcal{X}_M$} \}| = \binom{M^2}{n} = \frac{M^2 (M^2 - 1) (M^2 - 2) (M^2 - 3) \dots (M^2 - n + 1)}{n!}\]

Por otro lado, contemos las configuraciones en las que necesitamos menos de $\frac{n}2$ semirrectas para cubrir todos los puntos. Estas configuraciones obviamente están incluidas (de hecho son iguales) a las configuraciones en las que hay (al menos) 3 puntos alineados.

\[|\{\text{formas de elegir $n$ puntos de $\mathcal{X}_M$ tal que haya 3 alineados}\}| \leq \frac{M^2 (M^2 - 1) n (M^2 - 3) \dots (M^2 - n + 1)}{n!}\]

Esta fórmula vale porque $M^2 (M^2 - 1)$ son las formas de elegir los primeros 2 puntos. Luego tengo, como máximo, $n$ puntos que puedo elegir sobre la semirrecta formada por los primeros 2 puntos. Finalmente, elijo los $n - 3$ puntos restantes como quiera. Divido por $n!$ para eliminar las permutaciones.

Ahora, calculemos la probabilidad de que, dada una configuración al azar, esta requiera menos de $\frac{n}2$ semirrectas para cubrir todos los puntos. Llamaremos a esta probabilidad $p$.

\[p \leq \frac{|\{\text{formas de elegir $n$ puntos de $\mathcal{X}_M$ tal que haya 3 alineados} \}|}{|\{\text{formas de elegir $n$ puntos de $\mathcal{X}_M$} \}|} \leq \frac{n} {M^2 - 2}\]


Entonces, si tomamos los puntos sobre $\mathbb{N}^2$, como el problema indica y queremos calcular la probabilidad $p$, basta con hacer tender $M$ a $\infty$, que se ve claramente que tiende a 0.

Más aún, en nuestros casos experimentales, elegimos $M = n^2$, por lo tanto queda que

\[p \leq \frac{n} {n^2 - 2} \to 0\]

Confirmando formalmente lo observado experimentalmente.


